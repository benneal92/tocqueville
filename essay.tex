\documentclass{article}
\usepackage[utf8]{inputenc}
\usepackage{graphicx} % Add this line for images
\usepackage{hyperref}

\title{CLT Philosophy Meetup: Tocqueville}
\date{January 11 2026}

\begin{document}
\maketitle

\section{Disclaimer}  
Views expressed by the host are not representational of the group. 
Tocqueville views are to be worked through not simply agreed with.
Please feel free to challenge and provide alternative perspectives respectfully.

\section{Opener}
\textbf{Would you rather live in a society with perfect equality or perfect freedom?}

\section{Links}
  
\begin{itemize}
    \item \href{https://www.c-span.org/program/compiled-program/traveling-tocquevilles-america-video/115967}{Traveling Tocqueville's America}
    \item \href{https://www.youtube.com/watch?v=AXQvd_aPxog}{French Revolution}
\end{itemize}


\section{Two Countries}
\subsection{The French Revolution}
\begin{itemize}
    \item Began in 1789 driven by Enlightenment ideals of liberty, equality, and fraternity
    \item Radical phase (1792–1794): regicide, Reign of Terror, and mass executions
    \item Attempted top-down transformation through extreme centralization of power
    \item Ended in Napoleon's authoritarian rule, destroying the aristocratic order but failing to establish stable democracy
\end{itemize}
\subsection{France Post Revolution}
\begin{itemize}
    \item Post-revolutionary France was grappling with political instability and the dismantling of the aristocratic order
    \item Central question: Is democracy inevitable, and can it preserve liberty?
    \item The French Revolution had left a legacy of tension between equality and liberty often resulted in chaos or authoritarianism
    \item Fear that equality's advance might produce tyranny of the majority and crush individual freedom
\end{itemize}

\subsection{The American Revolution}
\begin{itemize}
    \item Sparked by colonial resistance to taxation without representation; Declaration of Independence (1776) asserted natural rights
    \item Military victory with French assistance (1783); Constitutional Convention (1787) created federal system with separation of powers
    \item Built upon existing local institutions and traditions rather than destroying them
    \item Established a stable republican government without radicalism or terror
\end{itemize}
\subsection{America Post Revolution}
\begin{itemize}
    \item 1830s America as a "living laboratory"—a society born democratic without an aristocratic past
    \item Functioning republic with widespread participation, voluntary associations, decentralized power, and local self-government
    \item Relative equality among white men, though Tocqueville noted the injustice of slavery and Native American treatment
    \item Liberty sustained through civic engagement, free press, and religion's moral influence, but threatened by conformism, materialism, and individualism
\end{itemize}

\section{Tocqueville's Biography}
\begin{itemize}
    \item French political philosopher and historian, born in 1805 to an aristocratic family that narrowly escaped the guillotine during the Revolution—both his parents were imprisoned and his great-grandfather executed
    \item Officially traveled to America (1831–32) to study the prison system, but his real purpose was to understand democracy's future and whether it threatened liberty
    \item Covered over 7,000 miles across America in nine months—traveling by steamboat, stagecoach, and horseback through wilderness, visiting 17 of 24 states
    \item Journeyed deep into the frontier in Michigan
    \item In Missippi he witnessed the forced removal of the Choctaw nation on the trail of tears via the Indian Removal Act by Andrew Jackson
    \item Traveled through Fayetteville, NC by stagecoach while observing the stark contrasts between Northern democracy and Southern slavery
    \item Wrote \textit{Democracy in America} in his late twenties—Volume I (1835) became an instant sensation across Europe, making him famous at age 30; Volume II followed in 1840 with deeper philosophical analysis
    \item He warned that democracy's greatest threat was not tyranny of kings but "tyranny of the majority" and the tendency toward individualism that weakens civic bonds
    \item Briefly served as foreign minister in 1849 before Napoleon III's coup ended his political career; spent final years writing \textit{The Old Regime and the Revolution} (1856) while battling tuberculosis
\end{itemize}

\section{Democracy in America and Tocqueville's Core Ideas}
\subsection{Mores, Virtue, and Bonds} 
\begin{itemize}
    \item \textbf{``Self-interest well understood''} -- Tocqueville's famous formulation: Americans practice virtue not from pure altruism but from enlightened self-interest; they see that serving others serves themselves
    \item \textbf{Mores more important than laws} -- The habits, customs, and beliefs of citizens matter more for democracy's survival than constitutional design
    \item \textbf{Religion serves politics} -- ``Most important is not that all citizens profess the true religion, but that they profess a religion''; religion provides shared moral foundations and restrains individualism
    \item \textbf{Separation of church and state} -- But religion thrives in America precisely because it stays out of politics; political separation enables religious influence on mores
    \item \textbf{Ceremonies and social forms} -- Rituals, courtesies, and ``dressing up'' create bonds and shared dignity; on pride: ``I would willingly trade several of our small virtues for this vice''
    \item \textbf{Democracy loosens social bonds but tightens natural ones} -- Equality weakens hierarchical ties but strengthens family and intimate relationships
    \item \textbf{Women as guardians of mores} -- Especially Protestant women in America; they shape moral culture through domestic influence and education
\end{itemize}

\textbf{Questions}
\begin{itemize}
    \item \textbf{What's one unwritten rule in American society that everyone seems to follow? Why does it persist without being law?}
    \item \textbf{How do mores sustain or undermine democracy? Can you think of examples of each?}
    \item \textbf{In an increasingly secularized world, what role does religion play in society? What (if anything) can replace its function?}
    \item \textbf{Can democratic citizens maintain civic virtue, or does equality inevitably lead to self-interest and apathy?}
\end{itemize}

\subsection{Equality vs Freedom}
\begin{itemize}
    \item \textbf{Tension between absolute equality and absolute freedom} -- Perfect equality would require restricting liberty; perfect liberty generates inequality
    \item \textbf{Equality as the driving principle} -- Democracy's defining feature; the passion for equality drives all else
    \item \textbf{Diminishing tolerance for inequality} -- As equality increases, even small inequalities become intolerable
    \item \textbf{Two faces of equality} -- Can seek to raise others up or bring others down; "equality in freedom or equality in servitude"
    \item \textbf{Leveling effects} -- Excellence reduced but baseline humanity raised; "extremes collapse"
\end{itemize}   

\subsection{Materialism and Restlessness}
\begin{itemize}
    \item \textbf{Market economy and material pursuit} -- Commercial society makes wealth the primary measure of success
    \item \textbf{Instability of conditions} -- In democracy, fortunes rise and fall; no one's position is secure
    \item \textbf{Industrial aristocracy} -- "A weak kind of aristocracy" emerges in factories; masters and workers have no lasting bonds
    \item \textbf{Perpetual restlessness} -- ``A vague fear of not having chosen the shortest road''; anxiety amid prosperity
    \item \textbf{Religion as brake on materialism} -- Only moral force strong enough to restrain acquisitiveness
\end{itemize}

\textbf{Questions}
\begin{itemize}
    \item \textbf{Does the pursuit of equality threaten liberty or excellence? Can we have all three?}
    \item \textbf{Have we achieved equality of conditions? What inequalities remain most visible or troubling?} 
    \item \textbf{Do you think Americans today are more individualistic or more community-oriented? What evidence do you see?}
    \item \textbf{What tendencies of liberalism promote individualism and isolation? Are these inevitable?}
\end{itemize}  

\subsection{Centralization of Power}
\begin{itemize}
    \item \textbf{Democratic tendency toward centralization} -- Democracy naturally tends toward administrative centralization
    \item \textbf{Weakness of individuals} -- Equality makes citizens weak and isolated as individuals; only the state appears strong enough to solve collective problems
    \item \textbf{Decay of intermediate institutions} -- Local institutions and intermediate bodies decay; citizens appeal constantly to central government
    \item \textbf{Self-reinforcing cycle} -- Centralization both cause and effect of majority tyranny
\end{itemize}

\subsection{Democratic Despotism (Soft Tyranny)}
\begin{itemize}
    \item \textbf{A new form of servitude} -- Paternal, mild, but degrading; ``an immense tutelary power'' that keeps citizens in perpetual childhood
    \item \textbf{Trading freedom for comfort} -- Citizens gradually surrender liberty for security and convenience; consent to their own subjection
    \item \textbf{Absolute but gentle power} -- ``Absolute, detailed, regular, provident, and mild''; degrades rather than torments
    \item \textbf{Passive citizenship} -- Citizens reduced to ``a flock of timid and industrious animals'' focused on private pleasures
    \item \textbf{Equality's role} -- Democratic peoples particularly susceptible due to love of equality and individualism; isolated individuals turn to the state
\end{itemize}

\subsection{Tyranny of the Majority}
\begin{itemize}
    \item \textbf{Omnipotence of the majority} -- In democracy, the majority holds absolute power—more total than any king; few legal safeguards against its will
    \item \textbf{Tyranny through social pressure} -- Operates through public opinion and conformity, not just laws; creates a ``moral empire'' over thought itself
    \item \textbf{Self-censorship and ostracism} -- ``You are free to think differently, but you are henceforth a stranger among your people''; social death replaces legal persecution
    \item \textbf{Intellectual mediocrity} -- Fear of standing out produces conformity of opinion; absence of great writers despite freedom
    \item \textbf{Danger increases with equality} -- Greatest where democratic equality is most complete
\end{itemize}

\textbf{Questions}
\begin{itemize}
    \item \textbf{Is social pressure a legitimate form of tyranny, or simply democracy in action?}
    \item \textbf{Do we see evidence today of citizens trading freedom for convenience or security?}
    \item \textbf{How do we distinguish between helpful government services and the kind of ``tutelary power'' Tocqueville warns against?}
    \item \textbf{Do modern welfare states represent Tocqueville's ``soft despotism,'' or do they enhance freedom?}
    \item \textbf{Do you think social pressure to conform is a serious threat to freedom today? How does it operate?}
    \item \textbf{Can you think of examples where majority opinion silences minority views—either through law or social pressure?}
\end{itemize}

\subsection{Remedies: Local Institutions and Associations}
\begin{itemize}
    \item \textbf{Township government} -- Provides practical school of self-governance; gives citizens taste of power and responsibility
    \item \textbf{Jury service} -- Educates people in rights and teaches practical judgment; makes every citizen a participant in governance
    \item \textbf{Voluntary associations} -- Counteract individualism and governmental centralization; citizens learn to cooperate
    \item \textbf{Direct participation} -- Local involvement prevents political apathy and teaches civic skills
    \item \textbf{Bulwark against tyranny} -- Decentralization serves as defense against tyranny of both majority and state
\end{itemize}

\subsection{Other Safeguards}
\begin{itemize}
    \item \textbf{Independent judiciary} -- Power of judicial review protects rights against majority will
    \item \textbf{Legal profession} -- Lawyers serve as an ``aristocratic element''; reverence for precedent and forms restrains democratic passion
    \item \textbf{Free press} -- Despite its own tendency toward conformity, serves as check on power
    \item \textbf{Religion} -- Moral check on materialism and majority power; provides shared moral foundations
    \item \textbf{Federal system} -- Dividing sovereignty prevents concentration of power
\end{itemize}

\textbf{Questions}
\begin{itemize}
    \item \textbf{Do you think there is a point where a political community become too large for meaningful self-governance?}
    \item \textbf{What role do local institutions play in preventing tyranny of the majority?}
    \item \textbf{What institutions or practices today might protect minority opinions from majority pressure?}
    \item \textbf{Are there new forms of association or institutions that might serve similar protective functions?}
\end{itemize}

\section{Bonus Material}
\subsection{Historical and Material Conditions} 
\begin{itemize}
    \item \textbf{Anglo-American inheritance} -- Americans brought English institutions and love of liberty
    \item \textbf{Puritan influence} -- Work ethic, local self-government, moral earnestness
    \item \textbf{No feudal past} -- America's exceptionalism: equality of conditions from the start
    \item \textbf{Frontier and land availability} -- Geographic expansion enabled independence and opportunity
    \item \textbf{Public education} -- Widespread literacy and civic education
    \item \textbf{Jacksonian democracy} -- Expansion of white male suffrage; rise of mass democracy
\end{itemize}

\textbf{Questions}
\begin{itemize}
    \item \textbf{Could another country copy the American government and get the same results? What would be required?}
\end{itemize}


\section{Tocqueville's Method}
Interdisplinary approach of History, Sociology, Political Science, and Philosophy

\subsection{Philosophical Approach}
\begin{itemize}
    \item Influenced by Montesquieu, Rousseau, and Pascal
    \item Emphasized learning politics through practice rather than abstract theory
    \item Skeptical of religious dogma and absolute philosophical truths
    \item Combined historical analysis with teleological thinking—sought first principles (especially equality) that drove social and political change
    \item Normative concerns: championed human flourishing, liberty, and excellence while lamenting the threat of mediocrity
\end{itemize}

\subsection{Sociological Approach}
\begin{itemize}
    \item Grounded in empirical observation and bottom-up analysis of social structures and relationships
    \item Macro-sociological focus: examined political institutions, civic associations, religion, and social equality
    \item Comparative method: contrasted American democracy with European aristocracy
    \item Cultural sensitivity: emphasized the role of customs, habits, and mores in shaping societies
    \item Psychological dimension: analyzed how democratic conditions shape character, desires, and behavior
\end{itemize}

\subsection{Criticisms of Tocqueville's Democracy in America}
\begin{itemize}
    \item \textbf{Exclusion of women and enslaved people} -- Tocqueville largely ignored the experiences of women and inadequately addressed slavery's centrality to American democracy
    \item \textbf{Racial blind spots} -- Failed to predict that race, not class, would become America's defining social cleavage
    \item \textbf{Overstated equality of conditions} -- Underestimated economic inequality and class stratification already present in 1830s America
    \item \textbf{Native American displacement} -- Witnessed Indian removal but treated it as tragic inevitability rather than democratic failure
    \item \textbf{Romantic view of townships} -- Idealized local democracy while overlooking conformity, exclusion, and majority tyranny at local levels
    \item \textbf{Limited economic analysis} -- Insufficient attention to capitalism's role in shaping democratic institutions and inequality
    \item \textbf{Deterministic framework} -- Tendency to present democratic trends as inevitable rather than contingent on political choices
    \item \textbf{Aristocratic assumptions} -- His fear of mass democracy and "tyranny of the majority" reflected elite anxieties about popular rule
\end{itemize}

\subsection{City of Charlotte Elected Positions}
\begin{itemize}
    \item \textbf{Mayor} -- Presides over City Council, serves as ceremonial head and city spokesperson; elected at-large to two-year terms
    \item \textbf{City Council Members (11)} -- Seven district representatives and four at-large members; legislative body that sets policy, approves budget, levies taxes, and appoints city manager
    \item \textbf{Mayor Pro Tem} -- Elected by City Council members to preside in mayor's absence and assume mayoral duties if needed
\end{itemize}

\subsection{Mecklenburg County Elected Positions}
\begin{itemize}
    \item \textbf{County Commissioners (9)} -- Six district representatives and three at-large members; govern county budget, property tax, and local priorities
    \item \textbf{Sheriff} -- Oversees law enforcement, jail operations, and court security for the county
    \item \textbf{District Attorney} -- Prosecutes criminal cases in the 26th Prosecutorial District
    \item \textbf{Clerk of Superior Court} -- Serves as probate judge; manages court records, jury management, and handles estates, guardianships, and adoptions
    \item \textbf{Register of Deeds} -- Maintains public records including property deeds, birth certificates, death certificates, and marriage licenses
    \item \textbf{Superior Court Judges} -- Hear felony criminal cases and civil cases over \$25,000
    \item \textbf{District Court Judges} -- Hear misdemeanor cases, civil cases under \$25,000, and family/juvenile matters
    \item \textbf{Soil and Water Conservation District Board} -- Manages local conservation programs and natural resource protection
\end{itemize}

\subsection{North Carolina State Elected Positions}
\begin{itemize}
    \item \textbf{Governor} -- Chief executive officer; prepares state budget, appoints department heads, can veto legislation
    \item \textbf{Lieutenant Governor} -- Presides over NC Senate, member of Board of Education and Community College Board, succeeds governor if needed
    \item \textbf{Attorney General} -- Heads Department of Justice; protects consumers, represents state agencies, supports local prosecutors
    \item \textbf{Secretary of State} -- Regulates businesses, charities, lobbyists, notaries; maintains state records
    \item \textbf{State Auditor} -- Oversees state financial accountability and conducts performance audits
    \item \textbf{State Treasurer} -- Manages state investments, pension funds, and financial operations
    \item \textbf{Superintendent of Public Instruction} -- Oversees K-12 public education system statewide
    \item \textbf{Commissioner of Agriculture} -- Regulates agricultural industry, food safety, and consumer protection
    \item \textbf{Commissioner of Labor} -- Enforces workplace safety, wage laws, and labor regulations
    \item \textbf{Commissioner of Insurance} -- Regulates insurance industry and protects insurance consumers
    \item \textbf{NC State Senate (50 seats)} -- Upper chamber of General Assembly; two-year terms
    \item \textbf{NC House of Representatives (120 seats)} -- Lower chamber of General Assembly; two-year terms
\end{itemize}

\section{Liberalism}
\subsection{Basics}

\begin{itemize}
    \item \textbf{Individualism}: The primacy of the individual over collective groups; society is composed of autonomous individuals with distinct interests
    \item \textbf{Liberty}: Commitment to personal freedom—civil liberties, freedom of speech, religion, and association are central
    \item \textbf{Equality}: Belief in equal moral worth of individuals, often expressed through equality before the law and political rights
    \item \textbf{Limited government}: State power should be constrained to protect individual rights, not dominate them
    \item \textbf{Rule of law}: Laws must apply equally to all, ensuring fairness and predictability in governance
    \item \textbf{Consent of the governed}: Legitimate authority derives from the consent of citizens, often through democratic participation
    \item \textbf{Private property}: Protection of property rights as a foundation for liberty and economic independence
    \item \textbf{Free markets}: Advocacy for free trade and minimal government interference in economic affairs
    \item \textbf{Rationalism}: Confidence in human reason and progress, rooted in Enlightenment ideals
    \item \textbf{Tolerance}: Respect for diversity of beliefs, lifestyles, and opinions within society
\end{itemize}

\subsection{Tradition}
\begin{itemize}
    \item \textbf{Thomas Hobbes (1588--1679)}
        \begin{itemize}
        \item \textit{Leviathan} (1651)
        \item \textit{De Cive} (On the Citizen) (1642)
        \item Main ideas: State of nature as war of all against all; social contract to escape chaos; absolute sovereignty necessary for peace; individuals as rights-bearers who consent to authority; foundation for modern political individualism
        \end{itemize}
    
    \item \textbf{John Locke (1632--1704)}
        \begin{itemize}
        \item \textit{Two Treatises of Government} (1689)
        \item \textit{An Essay Concerning Human Understanding} (1689)
        \item \textit{A Letter Concerning Toleration} (1689)
        \item Main ideas: Natural rights to life, liberty, and property; government by consent; right to revolution against tyranny; separation of church and state; limited government; property as foundation of liberty
        \end{itemize}
    
    \item \textbf{Montesquieu (1689--1755)}
        \begin{itemize}
        \item \textit{The Spirit of the Laws} (1748)
        \item Main ideas: Separation of powers; checks and balances; political liberty through institutional design; influence of climate and culture on government; distinction between types of regimes
        \end{itemize}
    
    \item \textbf{Jean-Jacques Rousseau (1712--1778)}
        \begin{itemize}
        \item \textit{The Social Contract} (1762)
        \item \textit{Discourse on Inequality} (1755)
        \item Main ideas: Popular sovereignty; general will; tension between natural freedom and civil society; critique of inequality and private property; direct democracy
        \end{itemize}
    
    \item \textbf{Adam Smith (1723--1790)}
        \begin{itemize}
        \item \textit{The Wealth of Nations} (1776)
        \item \textit{The Theory of Moral Sentiments} (1759)
        \item Main ideas: Free markets and division of labor; invisible hand; limited government intervention in economy; sympathy as basis of moral judgment; connection between commerce and liberty
        \end{itemize}
    
    \item \textbf{Immanuel Kant (1724--1804)}
        \begin{itemize}
        \item \textit{Groundwork of the Metaphysics of Morals} (1785)
        \item \textit{Perpetual Peace} (1795)
        \item Main ideas: Autonomy and human dignity; categorical imperative; republican government; cosmopolitan right; enlightenment as emergence from self-imposed immaturity
        \end{itemize}
    
    \item \textbf{The Federalist Papers (1787--1788)}
        \begin{itemize}
        \item Authors: Alexander Hamilton, James Madison, John Jay
        \item Main ideas: Republican government over large territory; extended republic to control factions; separation of powers and federalism; constitutional design to protect liberty while enabling effective governance
        \end{itemize}
    
    \item \textbf{Benjamin Constant (1767--1830)}
        \begin{itemize}
        \item \textit{The Liberty of the Ancients Compared with that of the Moderns} (1819)
        \item \textit{Principles of Politics} (1815)
        \item Main ideas: Distinction between ancient (participatory) and modern (individual) liberty; representative government suited to modern commercial society; protection of individual rights against majority tyranny
        \end{itemize}
\end{itemize}

\begin{figure}[h]
    \centering
    \includegraphics[width=0.5\textwidth]{images/Alexis_de_Tocqueville.jpg}
    \caption{Alexis de Tocqueville}
    \label{fig:tocqueville}
\end{figure}

\begin{figure}[h]
    \centering
    \includegraphics[width=0.5\textwidth]{images/Chateau.png}
    \caption{Chateau in Normandy France}
    \label{fig:Chateau}
\end{figure}

\begin{figure}[h]
    \centering
    \includegraphics[width=0.5\textwidth]{images/map.png}
    \caption{Travels}
    \label{fig:Map}
\end{figure}

% \begin{thebibliography}{9}
% \bibitem{source1}
% Alexis de Tocqueville, \emph{Democracy in America}, The University of Chicago Press, 2000, p.120
% \bibitem{source2}
% Cheryl B. Welch, \emph{The Cambridge Companion to Tocqueville}, cambridge university press, 2006, p.120 
% \end{thebibliography}

\end{document}